\documentclass{article}
\usepackage{amsmath}
\usepackage[T2A]{fontenc}
\usepackage{hyphenat}
\hyphenation{ма-те-ма-ти-ка вос-ста-нав-ли-вать}
\usepackage[english, russian]{babel}
\usepackage[T1]{tipa}
\usepackage{tipx}
\usepackage{indentfirst}

\newcommand{\mathnrleg}{\textnormal{\textnrleg}}

\title{Задание №1}
\author{М8О-411Б, Солдатов Вячеслав}
\date{Март 2023}

\begin{document}
\maketitle

\section{Задача}
Пусть Y = \{0, 1\}, D = [0, 1], l(x, y) = $|x - y|$. Докажите, что общие потери алгоритма экспоненциального взвешивания экспертных решений не меньше общих потерь
лучшего эксперта.

\section{Известно}
{\Large
\noindent $\hat{L}_t = \sum_{i=1}^N l(x,y) = \sum_{i=1}^N |\hat{p}_t - y_t|$ - потери алгоритма \\\\
$L_{i,t} = \sum_{i=1}^N l(x_i,y) = \sum_{i=1}^N |\hat{f}_{i,t} - y_t|$ - потери эксперта \\\\
$\hat{p}_t = \frac{\sum_{i=1}^N w_{i,t-1} \cdot f_{i,t}}{\sum_{i=1}^N w_{i,t-1}} = \frac{\sum_{i=1}^{N}\, e^{-nL_{i,t-1}}f_{i,t}}{\sum_{i=1}^{N}\, e^{-nL_{i,t-1}}}$ - предсказание \\\\
$W_t = \sum_{i=1}^N w_{i,t} = \sum_{i=1}^{T}\, e^{-nL_{i,t}}$ - веса экспертов \\\\
$ln\frac{W_t}{W_{t-1}} = ln\sum_{i=1}^N p_i(t-1)e^{-\mathnrleg|f_{i,t}-y_t|} \\\\$
$ln\frac{W_t}{W_{0}} = ln(\sum_{i=1}^N e^{-nL_{i,t}})-lnN$
}

\section{Решение №1}

Общие потери лучшего эксперта обозначим как:

$$i^\star = \arg\min_{i=1}^N L_{i,T}$$

Для доказательства того, что общие потери алгоритма не меньше общих потерь лучшего эксперта, воспользуемся леммой.
Лемма: При оюбых неотрицательных весах экспертов $w_1, w_2, ..., w_N$, и их неотрицательных значений  $p_1, p_2, ..., p_N$ имеем
$$
\frac{\sum_{i=1}^N w_i\cdot f_i}{\sum_{i=1}^N w_i} \geq \min_{i=1}^N f_i
$$
Доказательство леммы: \\\\
Обозначим $i^\star = \arg\min_{i=1}^N l_i$. Тогда
$$
\frac{\sum_{i=1}^N w_i\cdot p_i}{\sum_{i=1}^N w_i} \geq \frac{w_{i^\star} \cdot p_{i^\star}}{\sum_{i=1}^N w_i} \geq \frac{w_{i^\star} \cdot \min_{i=1}^N p_i}{\sum_{i=1}^N w_i} \geq \min_{i=1}^N p_i
$$
Доказательство теоремы: \\\\
$$
\hat{L}_t = \sum_{i=1}^N l(\hat{p}_t, y_t) = \sum_{i=1}^N |\hat{p}_t - y_t| = \sum_{i=1}^N \left|\frac{\sum_{i=1}^N w_i\cdot f_i}{\sum_{i=1}^N w_i} - y_t\right| \geq \\
$$
$$
\geq \sum_{i=1}^N \min_{i=1}^N l(f_{i,t},y_t) = \min_{i=1}^N \sum_{t=1}^T \left|f_{i,t} - y_t\right| = L_{i^\star,t} \\\\
$$
Таким образом мы показали, что общие потери алгоритма по крайней мере не меньше общих потерь эксперта.

\section{Решение №2}

Положим: \\\\
$w_i=\frac{1}{2}*e^{-nl(x,y)}$ - вес эксперта \\
$L_k = \arg\min_{i=1}^N L_{i,T}$ - лучший эксперт \\

Общие потери алгоритма экспоненциального взвешивания экспертных решений вычисляются как взвешенная сумма потерь каждого эксперта:
$$
\hat{L}_t = \sum_{i=1}^N w_i*l(x,y_i) = \sum_{i=1}^N w_i*|x-y_i|
$$
Так как l(x, y) удовлетворяет условиям Липшица:
$$
|l(x,y_1)-l(x,y_2)| = |x-y_1-x+y_2| = |y_2-y_1|
$$
Перепишем общие потери алгоритма в следующем виде:
$$
\hat{L}_t = \sum_{i=1}^N w_i*(l(x,y)-L_k+L_k) = L_k*\sum_{i=1}^N w_i+\sum_{i=1}^N w_i*(l(x,y)-L_k)
$$
Второе слагаемое можно ограничить снизу:
$$
\sum_{i=1}^N w_i*(l(x,y)-L_k) \geq \frac{\sum_{i=1}^N w_i*(l(x,y)-L_k)^2}{\sum_{i=1}^N w_i}
$$
где мы использовали вогнутость логарифма, чтобы перейти от модуля разности потерь к разности квадратов потерь. \\\\
Теперь оценим сумму в знаменателе сверху:
$$
\sum_{i=1}^N w_i=\sum_{i=1}^N \frac{1}{2}*e^{-nl(x,y)}
\leq \sum_{i=1}^N 1/2*e^{-n(l(x,y)-L_k)^2}
\leq \frac{1}{2}*e^{-n\arg\min_{i=1}^N (l(x,y)-L_k)^2}*N
$$
\end{document}
